% Remerciements / Acknowledgements
%
%  Grâce aux remerciements, l'auteur attire l'attention du lecteur
% sur l'aide que certaines personnes lui ont apportée, sur leurs
% conseils ou sur toute autre forme de contribution lors de la
% réalisation de son mémoire. Le cas échéant, c'est dans cette section
% que le candidat doit témoigner sa reconnaissance à son directeur de
% recherche, aux organismes dispensateurs de subventions ou aux
% entreprises qui lui ont accordé des bourses ou des fonds de
% recherche.
\ifthenelse{\equal{\Langue}{english}}{
	\chapter*{ACKNOWLEDGEMENTS}\thispagestyle{headings}
	\addcontentsline{toc}{compteur}{ACKNOWLEDGEMENTS}
}{
	\chapter*{REMERCIEMENTS}\thispagestyle{headings}
	\addcontentsline{toc}{compteur}{REMERCIEMENTS}
}
%


Je remercie vivement Amal Zouaq 
pour son encadrement exigeant mais bienveillant, et pour ses conseils toujours pertinents.

Merci également à Michel Gagnon, qui m'a accueilli très chaleureusement à Montréal et m'a fait découvrir le monde du Web sémantique.


Ce travail a été soutenu par le Fond d'excellence en recherche d'Apogée Canada. Les ressources de calcul provenaient de Calcul Québec\footnote{\href{www.calculquebec.ca}{www.calculquebec.ca}} et Calcul Canada\footnote{\href{www.computecanada.ca}{www.computecanada.ca}}.
