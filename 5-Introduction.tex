% Dans l'introduction, on présente le problème étudié et les buts
% poursuivis. L'introduction permet de faire connaître le cadre de la
% recherche et d'en préciser le domaine d'application. Elle fournit
% les précisions nécessaires en ce qui concerne le contexte de
% réalisation de la recherche, l'approche envisagée, l'évolution de
% la réalisation. En fait, l'introduction présente au lecteur ce
% qu'il doit savoir pour comprendre la recherche et en connaître la
% portée.
\Chapter{INTRODUCTION}\label{sec:Introduction}  % 10-12 lignes pour introduire le sujet.

Les graphes de connaissance constituent historiquement l'ossature du Web sémantique, et trouvent aujourd'hui des applications toujours plus variées : ils servent pour la recherche d'information \cite{bounhas2019building, dietz2018utilizing}, la recommandation de contenu \cite{ying2018graph, wang2018ripplenet, wang2019explainable}, la réponse automatique aux questions \cite{zhang2018variational, lukovnikov2017neural, saha2018complex}, mais aussi pour la recherche biomédicale \cite{bakal2018exploiting, sousa2020evolving}. Comme n'importe quelle base de données, un graphe de connaissance stocke de l'information de manière structurée et la rend accessible par le biais de requêtes; toutefois, contrairement aux bases de données relationnelles classiques, il ne nécessite pas de définir un schéma \textit{a priori} et une fois pour toutes. Cela facilite l'ajout de faits nouveaux, permet d'aggréger plus facilement des sources diverses et de lier des graphes de connaissance entre eux, ce qui constitue la clé du Web des données (\textit{Linked Data}).
Par rapport au texte, l'autre moyen dominant pour représenter la connaissance humaine, le graphe de connaissance présente l'avantage d'avoir une sémantique formelle, qui fournit une interprétation non-ambigüe des faits qu'il contient et permet l'inférence de faits nouveaux.




\clearpage

%%
%% ELEMENTS DE LA PROBLEMATIQUE
%%
\section{Éléments de la problématique}  % environ 3 pages




Présentation générale du Web sémantique et des graphes de connaissance : définition et applications

Ontologies et taxonomies : qu'est-ce qu'une ontologie ou un schéma; utilisations : prédiction de faits nouveaux, question answering, typage, recherche d'information

Construire ou compléter une ontologie : construction manuelle chère, impuissance des raisonneurs

Plongements de graphe



% \begin{table}[ht]
% \caption{Plages de valeurs pour le champ \texttt{DSCP}}
% \centering
% \begin{tabular}{|c|c|l|}
% \hline\rowcolor[gray]{0.8}\color{black}
% Plage & Valeurs & Règle d'assignation\\\hline
% 1 & xxxxx0 & Assignation par une norme de l'IANA\\\hline
% 2 & xxxx11 & Expérimentation/Usage local\\\hline
% 3 & xxxx01 & Expérimentation/Usage local (pourrait être jointe à la plage % 1)\\\hline
% \end{tabular}
% \label{tab:RangesDSCP}
% \end{table}

% On veut éviter que la figure et le tableau soient placés au-delà de la section courante.
% \FloatBarrier


%%
%% OBJECTIFS DE RECHERCHE / RESEARCH OBJECTIVES
%%
\section{Objectifs de recherche}  % 0.5 page


%%
%% PLAN DU MEMOIRE / THESIS OUTLINE
%%
\section{Plan du mémoire}  % 0.5 page


On présente d'abord une panorama de la littérature existante. Y sont introduit les concepts fondamentaux du Web sémantique : les graphes de connaissance, qui permettent une représentation structurée de la connaissance, et la logique descriptive, qui permet d'enrichir ces graphes de règles logiques dont la somme constitue une taxonomie ou une ontologie, selon leur complexité. On décrit ensuite les techniques existantes pour extraire automatiquement ces taxonomies, soit à partir d'un graphe, soit à partir de corpus textuels. On propose finalement un survol des méthodes pour inférer de nouveaux axiomes à partir d'un graphe.

Le chapitre \ref{chap:kge} est consacré aux modèles de plongement, qui sont au cœur du travail présenté ici et qui permettent une représentation vectorielle des éléments d'un graphe. On présente différents modèles concurrents, en s'efforçant de dégager les intuitions et les hypothèses qui ont dirigé leur conception, et de mettre en évidence leurs limitations théoriques et pratiques. On définit ensuite une nouvelle tâche pour l'évaluation de ces modèles, et on présente les résultats de cette évaluation.

Dans le chapitre \ref{chap:te}, on présente une nouvelle approche pour extraire automatiquement une taxonomie à partir des plongements d'un graphe de connaissance.

Le chapitre \ref{chap:texp} applique certaines des idées précédentes à l'extraction d'une taxonomie expressive. 
\clearpage