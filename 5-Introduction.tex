% Dans l'introduction, on présente le problème étudié et les buts
% poursuivis. L'introduction permet de faire connaître le cadre de la
% recherche et d'en préciser le domaine d'application. Elle fournit
% les précisions nécessaires en ce qui concerne le contexte de
% réalisation de la recherche, l'approche envisagée, l'évolution de
% la réalisation. En fait, l'introduction présente au lecteur ce
% qu'il doit savoir pour comprendre la recherche et en connaître la
% portée.
\Chapter{INTRODUCTION}\label{sec:Introduction}  % 10-12 lignes pour introduire le sujet.



\clearpage

%%
%% ELEMENTS DE LA PROBLEMATIQUE
%%
\section{Éléments de la problématique}  % environ 3 pages


\begin{table}[ht]
\caption{Plages de valeurs pour le champ \texttt{DSCP}}
\centering
\begin{tabular}{|c|c|l|}
\hline\rowcolor[gray]{0.8}\color{black}
Plage & Valeurs & Règle d'assignation\\\hline
1 & xxxxx0 & Assignation par une norme de l'IANA\\\hline
2 & xxxx11 & Expérimentation/Usage local\\\hline
3 & xxxx01 & Expérimentation/Usage local (pourrait être jointe à la plage 1)\\\hline
\end{tabular}
\label{tab:RangesDSCP}
\end{table}

% On veut éviter que la figure et le tableau soient placés au-delà de la section courante.
\FloatBarrier


%%
%% OBJECTIFS DE RECHERCHE / RESEARCH OBJECTIVES
%%
\section{Objectifs de recherche}  % 0.5 page


%%
%% PLAN DU MEMOIRE / THESIS OUTLINE
%%
\section{Plan du mémoire}  % 0.5 page

\clearpage