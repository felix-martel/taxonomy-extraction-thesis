\Chapter{EXTRACTION DE TAXONOMIE EXPRESSIVE}
\label{chap:texp}

\section{Motivation et principes généraux}

Dans la section précédente, on a décrit une méthode pour extraire une hiérarchie entre les types à partir des seuls plongements d'entité. 

Méthode précédente : on s'interdit d'utiliser le linked data => comment retrouver l'information connue (la taxonomie DBpédia) à partir des seuls plongements vectoriels ?

Cette méthode : en utilisant toute l'information accessible, comment aller plus loin que ce qui existe (en l'occurence, trouver une taxonomie expressive – qui à l'heure actuelle n'existe pas)

Pas d'utilisation du Linked Data = méthode très descriptive, qui n'utilise pas toute l'information à notre disposition.

Schéma général proche de la méthode précédente : regroupement hiérarchique sur les entités, puis transformation de la hiérarchie entre entités en une hiérarchie sur les classes. Ici, étiquettage de la classe plus sophistiqué, puisuq'on s'autorise Linked Data

Pourquoi ça marche ? Parce qu'on travaille sur un groupe d'entités dont on sait qu'elles sont sémantiquement proches, de par la géométrie de leurs plongements. Donc on restreint la dimension de l'espace de recherche, qui est un goulot d'étranglement habituel des méthodes d'extraction d'axiomes.

Toutefois, un ajout majeur : le retirage récursif des entités à regrouper pour limiter la propagation des erreurs dans l'arbre, limiter le bruit et affiner progressivement la spécificité des classes extraites. Plus un \textit{bag of tricks} pour que ça marche bien.

\section{Méthode proposée}
\subsection{Regroupement hiérarchique récursif avec retirage}

Rappeler le principe général

Exposer le principe de récursivité avec un schéma

Modalités de retirage

Expérimentation avec/sans retirage

Seuil adaptatif


\subsection{Extraction d'axiomes}
\label{subsec:texp-exaxiom}

Une fois que l'on dispose de clusters hiérarchisés, il reste à 
Qui peut être vu comme un étiquetage automatique des clusters

Idée générale

Extraction des atomes

Ici, nous proposons un schéma d'axiome extraction très simple, basé sur des statistiques d'occurrence au sein d'un cluster. Toutefois, la méthode peut s'adapter à beaucoup d'autres algorithmes d'extraction d'axiomes. 

\subsubsection{Couverture, spécificité et score de partition}
Soit $C = \{e_1, e_2, \ldots, e_n \} \subseteq \Ent$ un cluster contenant $n$ entités. Si $C$ n'est pas une feuille, alors il a deux sous-clusters gauche et droit, notés $L$ et $R$, et contenant respectivement $n_1$ et $n_2$ entités, avec $n = n_1 + n_2$. En notant $\sqcup$ l'union disjointe, on a donc :
\begin{equation}
C = L \sqcup R
\end{equation}
Pour un axiome logique $A$ et une entité $x \in \Ent$, on note $A(x)$ si $x$ vérifie l'axiome $A$. On se propose d'expliquer la séparation du cluster $C$ en ses deux sous-clusters, c'est-à-dire d'identifier des axiomes qui sont vrais dans l'un des clusters mais pas dans l'autre. Ce choix a d'abord été fait dans le but de mieux comprendre le fonctionnement du regroupement hiérarchique et d'analyser l'arbre de clustering obtenu. Toutefois, il est apparu que cette approche pouvait servir également à étiquetter des clusters, et donc à leur attribuer des axiomes.

Expliquer la division de $C$ en $L \sqcup R$ nécessite de trouver un axiome $A$ tel que $A$ est valide pour tous les éléments de $L$ et pour aucun élément de $R$ :
\begin{equation}
    \left(\forall x \in L, A(x)  \right) \land \left(\forall x \in R, \neg A(x) \right)
\end{equation}
Soit, de manière équivalente :
\begin{equation}
    \forall x \in C = L \sqcup R, A(x) \oplus (x \in R)
    \label{eq:exaxiom-xor-def}
\end{equation}
Avec $\oplus$ l'opérateur «ou exclusif». L'équation \ref{eq:exaxiom-xor-def} signifie qu'une entité de $C$ ne peut pas à la fois être dans $R$ et vérifier $A$, et elle ne peut pas non plus être dans $L$ sans vérifier $A$. Cette équation correspond au cas optimal où il existe un axiome qui divise parfaitement $C$ en $L$ et $R$ : en pratique, la plupart des clusters ne peuvent pas être parfaitement divisés, et il nous faut donc mesurer à quel point on se trouve de l'optimalité. Pour cela, on commence par définir la \textit{précision} d'un axiome $A$ par rapport à un ensemble d'éléments $E \sqsubset \Ent$ quelconque comme la proprtion d'éléments de $E$ qui vérifient $A$ :
\begin{equation}
    \text{prec}(A, E) = \frac{|\{ x \in E, A(x)\}|}{| E |}
\end{equation}
On mesure alors la capacité d'un axiome $A$ à expliquer la division $C = L \sqcup R$ avec deux métriques : d'une part, sa \textit{couverture}, définie comme la proportion d'éléments de $L$ qui vérifient $A$; d'autre part, sa \textit{spécificité}, qui indique la proportion d'éléments de $R$ qui ne vérifient pas $A$. Ces deux métriques se calculent à partir de la précision comme suit :
\begin{align}
    \text{cov}_{L \sqcup R}(A) &= \text{prec}(A, L) \\
    \text{sep}_{L \sqcup R}(A) &= 1 - \text{prec}(A, R)
\end{align}
On combine ces deux mesures en un seul indicateur synthétique, que l'on appelle le \textit{score de partition} de $A$, à l'aide d'une moyenne harmonique :
\begin{equation}
    \text{part}_{L \sqcup R}(A) = \left(  \text{cov}_{L \sqcup R}(A)^{-1} + \text{sep}_{L \sqcup R}(A)^{-1} \right)^{-1}
\end{equation}

On peut vérifier que l'on retrouve bien l'intuition derrière l'équation \ref{eq:exaxiom-xor-def}. La proportion d'éléments qui vérifient la condition de séparation \ref{eq:exaxiom-xor-def} est donnée par :
\begin{equation}
    \text{xor}(A) = \frac{| \{x \in C : A(x) \oplus (x \in R) \}|}{| C |}
\end{equation}
Par définition de l'opérateur ou exclusif, on peut écrire :
\begin{equation}
    n \cdot \text{xor}(A) = |\{x \in C: (A(x) \land \neg (x \in R) \lor (x \in R \land \neg A(x)) \}|
\end{equation}
Puis, comme $C$ est l'union disjointe $L$ et $R$, si $x \in C$, alors $\neg (x \in R) = x \in L$ et on a donc :
\begin{align*}
    n \cdot \text{xor}(A) &= |\{x \in L: (A(x)\} \sqcup \{x \in R : \neg A(x)) \}| \\
    &= |\{x \in L: (A(x)\} | + | \{x \in R : \neg A(x)) \}|  \\
    &= |\{x \in L: (A(x)\} | + | \{x \in R \} \setminus \{x \in R : A(x)) \}| \\
    &= n_1 \cdot \text{prec}(A, L) + n_2 - n_2 \cdot \text{prec}(A, R) \\
    &= n_1 \cdot \text{cov}(A) + n_2 \cdot \text{spe}(A) \\
\end{align*}
Soit finalement :
\begin{equation}
    \text{xor}(A) = \frac{n_1 \cdot \text{cov}(A) + n_2 \cdot \text{spe}(A)}{n_1 + n_2}
\end{equation}

Notation\todo{Déplacer ceci} : pour un axiome $A$ et un ensemble d'entités $E \sqsubset \Ent$, on note $A(E)$ l'ensemble des éléments de $E$ qui vérifient $A$ :
$$
A(E) = \{ x \in E : A(x) \}
$$
On vérifie facilement que les propriétés suivantes sont vraies :
\begin{align}
    & \top(E) = E  \\
    & \bot(E) = \varnothing  \\
    & (A \land B)(E) \subseteq A(E) \label{eq:prop-ax-and} \\
    & A(E) \subseteq (A \lor B)(E)  \\
    & E \subset E' \implies A(E) \sqsubset A(E') 
\end{align}
Et on a :
\begin{align}
    \pre_{L \sqcup R}(A) &= \frac{|A(L) |}{|L|} \\
    \rec_{L \sqcup R}(A) &= 1 - \pre(A, R) \\
\end{align}

Avant de présenter l'extraction d'axiome proprement dite, relevons quelques propriétés de ces métriques. Soit $A, B$ deux axiomes. Alors :
\begin{align}
\cov(A \land B, E) &= \frac{|(A \land B)(L) |}{| L |} \\
                &\leq \frac{|A(L) |}{| L |} \text{ d'après l'équation \ref{eq:prop-ax-and}} \\
                &\leq \cov(A) \label{eq:prop-cov-land}
\end{align}
Il suit que :
\begin{align}
\cov(A \lor B) &= \frac{|\{ x \in L : A(x) \lor B(x) \}|}{| L |} \\
  &= \frac{|\{ x \in L : A(x) \}| + |\{ x \in L :  B(x) \} - |\{ x \in L : A(x) \land B(x) \}||}{|E|} \\
  &= \cov(A) + \cov(B) - \cov(A \land B)
\end{align}
Or, d'après la relation \ref{eq:prop-cov-land}, $\cov(B) \geq \cov(A \land B)$, d'où finalement :
\begin{equation}
    \cov(A \lor B) \geq \cov(A)
\end{equation}
Or, comme $\spe(x) = 1 - \cov(x)$, on peut obtenir les relations suivantes :
\begin{align}
    \spe(A \land B) &\geq \spe(A) \label{eq:prop-spe-land} \\
    \spe(A \lor B) &\leq \spe(A) 
\end{align}

\subsubsection{Construction d'axiomes complexes par améliorations successives d'axiomes atomiques}

On dispose d'un moyen pour évaluer la capacité d'un axiome $A$ à expliquer la partition d'un cluster en deux sous-clusters. Il reste à définir une procédure pour construire de tels axiomes. Pour cela, on définit d'abord des types d'axiomes primitifs, appelés des \textit{axiomes atomiques} ou simplement \textit{atomes} dans la suite; on extrait, pour chaque cluster, une liste d'axiomes atomiques, puis on combine ces atomes au moyen de conjonctions et de disjonctions pour produire des axiomes plus complexes.

On considère les types d'axiomes atomiques suivants :
\begin{itemize}
    \item les classes $C$, aussi appelés concepts en logique descriptive, comme par exemple \dbo{Agent}, \dbo{Person} ou \dbo{Place};
    \item les restrictions de la forme $\exists R.C$ avec $C$ une classe, par exemple $\exists \dbo{locatedIn}.\dbo{Country}$, qui contient toutes les entités situées dans un pays ;
    \item les restrictions de la forme $\exists R.\{v\}$ avec $v \in \Ent$ une entité, tel que $\exists \dbo{locatedIn}.\{\dbr{Canada}\}$ pour représenter l'ensembles des entités situées au Canada;
    \item les restrictions de la forme $\exists R.t$, avec $t$ représentant les littéraux d'un type $t$ donné, comme par exemple $\exists \dbo{birthDate}.\texttt{xsd:date}$ l'ensemble des entités dont la date de naissance est du type \texttt{xsd:date};
\end{itemize}

Pour chaque entité $x$ du cluster d'entrée, on extrait l'ensemble des triplets $(x, r, y)$ dont $x$ est le sujet. Si $r$ est la relation \rdf{type}, alors $y$ représente une classe, et le triplet $(x, r, y)$ est transformé en l'axiome atomique $y$. Si $y$ est un littéral, on extrait son type $t$, et on obtient l'axiome atomique $\exists r.t$. Autrement, on liste les classes $C_1, C_2, \ldots, C_m$ dont $y$ fait partie, et on extrait les axiomes atomiques $\exists r.\{y\}, \exists r.C_1, \ldots, \exists r.C_m$. On obtient ainsi une liste d'axiomes atomiques $\mathcal{A}_\text{atomes}$ pour l'entièreté du cluster; cette liste est filtrée et seuls les atomes vérifiés par plus de $\delta_\text{filtre} = 10\%$ des entités du cluster sont conservés.

Ensuite, on génère une liste d'axiomes candidates $\cal{A}_\text{candidats}$ à partir de cette liste d'axiomes atomiques. Initialement, les axiomes candidats sont simplement les axiomes atomiques. On évalue chaque axiome $a$ parmi les candidats, en calculant sa couverture, sa spécificité et son score. On compare alors ce score à un seuil $\delta$, par exemple $\delta = 0.9$. Si $\text{cov}(a) < \delta$, l'axiome $A$ ne couvre pas assez d'entités : on l'améliore alors itérativement en ajoutant des clauses OU. Pour chaque axiome atomique $b$, on génère un nouvel axiome candidat $a \lor b$, et on l'ajoute à la liste des axiomes candidats. D'après l'équation \ref{eq:prop-cov-land}, $\cov(a \lor b) \geq \cov(a)$.
À l'inverse, si $\spe(a) < \delta$, alors l'axiome est insuffisamment spécifique : il est vérifié par trop d'éléments de $R$. Suivant l'équation \ref{eq:prop-cov-land} : $\forall b, \spe(a \land b) \geq \spe(a)$, on peut améliorer cette spécificité en ajoutant une conjonction. Pour tout axiome atomique $b$, on génère l'axiome $a \land b$ et on l'ajoute à la liste des axiomes candidats. 
Si on a à la fois $\spe(a) < \delta$ et $\cov(a) < \delta$, alors l'axiome $a$ ne permet pas d'expliquer la partition $C = L \sqcup R$ et il est retiré de la liste des candidats. Enfin, si $\cov(a) > \delta$ et $\spe(a) > \delta$, l'axiome est conservé tel quel.

À chaque itération, on limite le nombre de candidats qui sont améliorés (c'est-à-dire étendus par des disjonctions ou des conjonctions) : seuls les $N_\text{ax}$ axiomes candidats avec le plus haut score de partition sont conservés, avec $N_\text{ax}$ un paramètre fixé empiriquement, typiquement $N_\text{ax} = 10$. La recherche d'axiomes s'arrête lorsqu'il n'y a plus d'axiomes candidats à améliorer ou lorsque le nombre d'itérations dépasse une certaine limite $N_\text{iter}$.

Le résultat de cet algorithme est un ensemble $\cal{A}(L)$ (éventuellement vide) d'axiomes capables de qualifier le sous-cluster $L$ par opposition au sous-cluster $R$, avec les scores associés. Si $\cal{A}(L) = \varnothing$, aucun axiome satisfaisant aux critères n'a été trouvé, et le cluster $L$ n'est donc pas étiquetté. Sinon, on étiquette $L$ avec l'axiome de plus haut score :
\begin{equation}
    \alpha^*(L) = \argmax_{a \in \cal{A}(L)}(\xscore(a))
\end{equation}


Combinaison d'atomes avec négation, conjonction, disjonction pour produire des axiomes

\todo{Mentionner l'autre idée (TF-IDF) ?} %Autre principe : TF-IDF based


\subsection{Algorithme général}
% Raffinements
\subsubsection{Vue d'ensemble}

Extraction de taxonomie à partir des axiomes

Choix des seuils

Discussion sur les paramètres

\subsubsection{Diminution du seuil}
 % Déplacer en section 5.2.1 ?

\subsubsection{Phase finale}

Phase finale : ajout des classes manquantes (si disponible), restriction des patterns


\section{Évaluation et discussion}






