\chapter*{NOTATIONS}
%

\section*{Ensembles}
\begin{itemize}
    \item $\bb{N}$ : l'ensemble des entiers naturels, $\bb{N} = \{1, 2, 3, \ldots \}$.
    \item $\R$ : l'ensemble des nombres réels.
    \item $\bb{C}$ : l'ensemble des nombres complexes.
    \item $\R^d$ (resp. $\mathbb{C}^d$) : l'ensemble des vecteurs réels (resp. complexes) de dimension $d$.
    \item $\R^{m \times n}$ (resp. $\bb{C}^{m \times n}$) : l'ensemble des matrices réelles (resp. complexes) de dimension $m$ par $n$.
 \end{itemize}
 
 \section*{Fonctions}
 \begin{itemize}
    \item $\lfloor x \rfloor$ : le plus grand entier inférieur ou égal à $x$.
     \item $\Re(z)$ : la partie réelle d'un nombre complexe $z \in \bb{C}$.
     \item $\Im(z)$ : la partie imaginaire d'un nombre complexe $z \in \bb{C}$.
 \end{itemize}
 
 \section*{Vecteurs, matrices}
 
 Dans tout le document, les caractères gras minuscules ($\bf{e, u, \ldots}$) désignent des vecteurs et les caractères gras majuscules ($\bf{A, M, X, \ldots}$) désignent des matrices.
 
 \begin{itemize}
    \item $\bf{I_d}$ : la matrice identité de dimension $d \times d$.
    \item $\bf{0_d}$ : le vecteur nul de dimension $d$.
    \item $\bf{0_{d \times d}}$ : la matrice nulle de dimension $d \times d$.
    \item $\bf{M}^\top$ : la transposée de la matrice $\bf{M}$.
    \item $\compconj{\bf{M}}$ : la matrice conjuguée d'une matrice complexe $\bf{M}$.
    \item $\diag(d_1, d_2, \ldots, d_k)$ : la matrice diagonale de dimension $k \times k$, dont la $i$-ème coordonnée diagnoale vaut $d_i$.
\end{itemize}

\section*{Graphes de connaissance}

\begin{itemize}
    \item $\Ent(G)$ : l'ensembles des entités (ou \textit{sommets}) d'un graphe $G$.
    \item $\Rel(G)$ : l'ensemble des relations (ou \textit{arêtes}) d'un graphe $G$.
    \item $\rel{h}{r}{t}$ : la propriété «$h$ est lié à $t$ par la relation $r$».
\end{itemize}