\Chapter{Contexte et travaux connexes}\label{chap:revue}


\section{Web sémantique et logique descriptive}
\label{sec:dl}

\subsection{Graphes de connaissance}

%Un graphe de connaissance permet de représenter des données de manière structurée. Il consiste en un ensemble d'objets du monde réel, appellés \textit{entités}, reliés entre eux par des \textit{relations}.

%On se donne $\Ent$ un ensemble d'entités, et $\Rel$ un ensemble de relations. En pratique, chaque entité ou relation est représentée par un identifiant unique, l'URI (\textit{Universal Resource Identifier}).

%Un graphe de connaissance 

Un graphe de connaissance est une collection structurée de données permettant de représenter des informations, ou des \textit{faits}, sur le monde réel. Le terme «graphe de connaissance» est un terme générique dont la définition varie d'un auteur à l'autre \cite{ehrlinger2016towards} : il est parfois utilisé comme synonyme d'un graphe RDF
- graphe RDF
- inclut un mécanisme de raisonnement
- restreint à un sous-domaine (WordNet)
- générique (YAGO)


Dans ce travail, on adopte une définition très générale d'un graphe de connaissance vu comme un ensemble d'entités liées entre elles par des relations. Pour un ensemble d'entités $\Ent$ et un ensemble de relations $\Rel$, un graphe de connaissance est simplement un ensemble de triplets $\KG \subseteq \Ent \times \Rel \times \Ent$. Si $(h, r, t)$ est un triplet du graphe, alors l'entité $h$ est reliée à l'entité $t$ par la relation $r$; dans ce cas, on dit que $h$ est le \textit{sujet} du triplet, et $t$ son objet. Le choix des notations s'explique par l'anglais, $h$ désignant la tête (\textit{head}) et $t$ la queue (\textit{tail}) du triplet. 
% Pour un graphe de connaissance, on parle fréquement de données \textit{multi-relationnelles} pour indiquer que les entités

Le format dominant pour représenter un tel graphe est le format RDF, pour \textit{Resource Description Framework} \cite{cyganiak14}. En RDF2, les entités sont de deux types : soit des identifiants, soit des littéraux. Les identifants, ou URI, pour \textit{Uniform Resource Identifier}, sont simplement des chaînes de caractères qui désignent des ressources du monde réel : lieux, personnes, documents, etc. %Il est courant de regrouper des identifiants au sein d'un \textit{vocabulaire} commun



Les littéraux permettent de représenter des valeurs fixes – nombres, textes, booléens, dates, sommes d'argent – 

% RDF

% Typologie des relations


\subsection{Ontologies et logique descriptive}

% Présentation

% Sémantique ?

% OWL

% Plongements de graphe

\section{Extraction automatique d'ontologie et de taxonomie}

% Raisonneurs

% Apprentissage statistique

% À partir de textes

% Class-based

% TIEmb / NICKEL

\section{Regroupement hiérarchique sur les graphes}
