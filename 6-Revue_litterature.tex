\Chapter{REVUE DE LITTÉRATURE}\label{chap:revue}
\section{Web sémantique et logique descriptive}
\label{sec:dl}

\subsection{Graphes de connaissance}

Un graphe de connaissance permet de représenter des données de manière structurée. Il consiste en un ensemble d'objets du monde réel, appellés \textit{entités}, reliés entre eux par des \textit{relations}.

On se donne $\Ent$ un ensemble d'entités, et $\Rel$ un ensemble de relations. En pratique, chaque entité ou relation est représentée par un identifiant unique, l'URI (\textit{Universal Resource Identifier}).

Un graphe de connaissance 

\section{Extraction automatique d'ontologie et de taxonomie}

\section{Regroupement hiérarchique sur les graphes}
